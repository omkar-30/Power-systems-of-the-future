\justifying

\section*{\textbf{Exercise 1: }}
\begin{itemize}
    \item HVDC converter stations generate characteristic and non-characteristic harmonic currents. For a twelve-pulse converter, the characteristic harmonics are of the order n = (12 * k) $\pm$ 1 (k = 1,2,3 ...). The purpose of the filter circuit is to provide sufficiently low impedances for the relevant harmonic components in order to reduce the harmonic voltages to an acceptable level. The AC filter is used to to absorb harmonic currents generated by the HVDC converter and thus to reduce the impact of the harmonics on the connected AC systems, like AC voltage distortion and telephone interference and to supply reactive power for compensating the demand of the converter station.
    \item Thyristors are used as switches and thus the valve becomes controllable. The thyristor valves are used to convert from AC to DC and hence form the central component of the HVDC converter station.  
    \item The 12-pulse converter has two 3-phase systems which are spaced apart from each other by 30 or 150 electrical degrees. This is achieved by installing a transformer on each network side in the vector groups Yy0 and Yd5. At the same time, they ensure the voltage insulation necessary in order to make it possible to connect converter bridges in series on the DC side, as is necessary for HVDC technology. The transformer main insulation, therefore, is stressed by both the AC voltage and the direct voltage potential between valve-side winding and ground. The converter transformers are equipped with on-load tap-changers in order to provide the correct valve voltage.
    \item  The DC filter is used to limit the voltage level and the transient stresses on the active part, so that comparatively low equipment ratings can be used. Appropriate design allows the exploitation of the positive characteristics of both passive and active filters.
    \item The surge arrester is to protect the equipment from the effects of overvoltages. During normal operation, it should have no negative effect on the power system. Moreover, the arrester must be able to withstand typical surges without incurring any damage.
\end{itemize}
\section*{\textbf{Exercise 3: }}
Connecting the wind farm to the shore will be done via cables going through the sea. Hence it requires a choice of cables. Furthermore type of transmission system is also a concern.There are indeed some options that can be considered. For instance, an XLPE cable or the lapped thin film insulation cable might be competitive choices.
There can be two scenarios, when the transmission is AC or it is HVDC. Both these cases are analysed further.
\section{AC Transmission:}
\begin{itemize}
    \item Having AC power transfer would obviously include transformers which allows to achieve different levels of transmission voltage. As the output of the wind farm will be AC, it will be fairly easy to set up this system without incurring certain converter costs. The important consideration is the placement of  the transformers, i.e. at the wind farm or on shore at the local grid.  Both configurations have their own pros. Keeping the transformers at the wind farm would ensure lower losses during transmission whereas at the local grid would mean that less switch-gear required at the wind farm.
    \item Another consideration is power loss happening in the cable. The loss across the cable is given by the following formula:
\[ P=I^{2}*R\]
 where I is the current through the cable and R is the resistance of the cable. Transformers make it possible, to step up/down the voltage by keeping the current low, thus reducing the losses. Whereas, the current level in HVDC transmission is significantly higher which can lead more losses across the cable.
 
 \item AC transmission requires frequency stabilization as well as reactive power compensation. This makes control more complicated as compared to HVDC transmission where everything is electronic. 
 
 \item Another approach that can be implemented is LFAC (Low Frequency AC) transfer[1]. This approach is currently investigation. The prime objective is to transfer AC power at a lower frequency of 50/3 Hz. 
 This requires a frequency converter at the local grid.
 
\end{itemize}

\section{HVDC Transmission:}
\begin{itemize}
    \item Electronic Power Conversion of AC to DC eliminated the possibilities of using transformer within the transmission circuit. Instead, it involves AC/DC and DC/AC converters. These converters help in achieving  desired level of transmission voltage with proper control. Furthermore, the losses are very less in this type. Losses are quoted to be about 3 percent for  1000km. 
    \item The cost of cable will eventually be the same as the distance of the wind farm to UK or the Netherlands is more than the break even distance of a submarine cable (~50 km). 
    \item Lower noise as compared to AC transmissions is also beneficial in DC power transfer.
    \item The only issue is availability of protective devices (e.g. circuit breakers) which are mostly AC. Having DC protective devices is a challenging task that is being studied and investigated.
\end{itemize}
Based on the summarized comparison, HVDC transmission prove to be more effective mode for power transfer. 
The following layout can be suggested for the given exercise:
\begin{itemize}
    \item The type of the turbine depends upon the capacity factor. As offshore wind farms have a capacity factor of ~0.5, turbines of greater efficiency will be required. A horizontal axis wind turbine(HAWT) can be a good option. 
    \item The layout can be mono-polar or a bipolar HVDC. Bipolar has an advantage of have two different voltage levels i.e. positive as well as negative levels. Advantages of bipolar network, is that in an event of the fault, one part of the network still supplies. The components will also include converters to convert the AC power generated to DC. The preferred level of DC transmission is 250 KV.
    \item As the transmission lines will run thorough water, the cable can be a XLPE cable. The cost of overall system may increase but so will the reliability.
    \item The overall cost of the system will be high depending upon the choice of the cable and the overall layout. The power conversion devices will also add to the cost. 
    \item Building a new HVDC connection will aid the current connection between the two countries. The existing inter-connector is a 1000 MW. Adding one more connection will increase the overall capacity as well as the reliability of the system. 
    \item Looking at the advantages of HVDC transmission, it is a possibility that the future grid will be DC. This makes the choice of transmission more relevant. 
\end{itemize}
\section{\textbf{References:}}
\begin{enumerate}
\item "Low frequency AC for offshore wind power transmission - prospects and challenges - IET Conference Publication", Ieeexplore.ieee.org, 2019. [Online]. Available: https://ieeexplore.ieee.org/document/7140551. [Accessed: 29- May- 2019].
\end{enumerate}