\justifying

\section{\textbf{Case 1}}
\begin{itemize}
    \item Scenario 1:  Request by the consumer for a new connection in the grid.
    \\In this scenario, all the actors in the system are affected. The increase in the demand keeping the generation constant creates imbalance in the system.  In terms of physical aspects, the system operator has to make sure that the frequency is stabilised and the extra demand is met. Also,  the access to and connection in the grid to the consumer is provided by the grid operator. 
    \item Scenario 2: :  Request by the generator for a new connection in the grid.
	\\Here, the generation increases but the demand is constant. The user is not affected implicitly   but rather explicitly as the price of the commodity (i.e. electricity ) decreases.  In order to avoid congestion, the system operator has to make sure that extra cables are laid which increases the CAPEX  and the OPEX of the grid operator.
\end{itemize}



\section{\textbf{Case 2:}}
An operator faces congestion on a line: more access right sold and used for generation than the line can continuously transport (n-1) securely.\\
In this case, the congestion on the line is caused primarily due to a fault in the system. The fault can be caused either due to environmental factors like storm or due to overheating of the line. \\
The actors that get affected are the grid operator, the user and the system operator. The options are described for the actors below: 

\begin{enumerate}
    \item Options for the operator:
    \begin{itemize}
        \item During an event of congestion, the system operator has to buy the reserves to ensure the supply is maintained. Due to this, the system operator has to incur cost of buying the reserves from BSP/BRPs (opex). 
        \item Another possible option for the operator to relieve the congestion is to shed the load so as to make sure the fault does not have cascaded effects and ensure the system is isolated from the fault. This may affect the user as the supply is interrupted and the user is inconvenienced greatly. The grid operator also has to incur the cost to relieve the fault which can be replacing of faulty component or techinician cost (capex).
    \end{itemize}
    \item Options for the User:
    \begin{itemize}
        \item To ensure there is uninterrupted supply of electricity, the user can buy capacity credits (Future and Options) so that in an event of disruption the system operator has to deliver the user to adhere to the agreement. In this situation, the user pays a premium over the base cost of electricity to prepare for unforeseen situations and also the system operator has to incur cost for buying the reserves from BSP/BRPs.
        \item The user can also make use of its personal storage for its own purposes. The user (as a prosumer) can also take part in relieving congestion by feeding electricity to the grid. In this case, there is capital cost for the user for storage (batteries) and integrating renewables (Solar, Wind). The grid operator has to ensure the line capacity is sufficient enough to allow the integration of DERs and therefore has to upgrade the infrastructure (capex).  
    \end{itemize}
\end{enumerate}

\section{\textbf{Case 3:}}
There can be two scenarios based on the problem statement.
The first situation would be caused when all the EV's are charging simultaneously which increases the  load on the grid. It can trigger an under frequency in the grid.
The second scenario where there is excessive uncontrolled generation( solar,wind) can lead to order frequency.
The options then available for the system operator are to dispatch the reserve (BSP) in order to cater to the increased demand, and if the dispatch fails to satisfy the demand then the last option left for the operator would be to shed load to maintain the integrity of the grid.In case of reserve being dispatched the consumer faces the risk of high pricing and lower quality of power, whereas if the TSO has to shed load then it is liable to pay the cost of volume of lost load.
Another novice solution to this would be to develop a demand side response in order to reduce the congestion of the grid. If multiple EV's are charging at time, then the control algorithm  deployed for demand side management can curb the power delivered to batteries which are sufficiently charged (within the guidelines agreed upon by the consumer).
The consumer can always enter into a long term contract with the TSO to assure the availability of power, in turn facilitating the building of extra capacity.
For the consumers in the both scenarios they can  benefit from having storage facilities locally. So whenever there is a over frequency, the storage can be charged using the excess electricity and in the case of under frequency the storage can supply electricity to the grid to relieve its congestion.Though this will affect the consumers capex cost.
The EV's can also provide backup power during the peak demand through the Vehicle to grid(V2G) technology.
A microgrid solution can also be applied in order to relieve the grid from congestion, by operating the microgrid in an island mode.

The grid operator in both the scenarios has to ensure that the phsical infrastructure is capable of handling the power flows, thereby increasing its capex costs. 

\section{\textbf{Case 4:}}
\textit{The power system of today cannot cope with this huge amount of variable and increasingly distributed electricity sources to guarantee the security  of supply (SoS). More flexibility is required}\\ \\
With the growing needs of electricity and the requirement for reducing the dependency on carbon-based fuels to transition towards a cleaner, greener and a sustainable future has compelled us to find sophisticated solutions in developing the grid infrastructure and design frameworks to benefit us.
\\
Some of the system and network needs are :
    \begin{itemize}
        \item Due to the fluctuating nature of renewable resources there is imbalance in the supply and this can be solved by a decentralised storage system to ensure stability of supply and in turn making the grid more reliable. 
        \item Microgrids are a good way to encourage local generation and also promotes decentralised control architecture. 
        \item The introduction of ICT in the grid infrastructure will enable autonomous operation thereby reducing the need for human involvement and also faster response to events such as faults in the system making it more reliable. 
        \item Standardisation of communication protocols, hardware and software architecture to allow inter-operability of devices. 
    \end{itemize}

\section{\textbf{Case 5:}}
For a reliable, sustainable and affordable power system the following solutions can be implemented:
\begin{enumerate}
    \item Technology based solutions:
    \begin{itemize}
        \item Energy Trading between micro-grids (user-user trading).
        \item Setting up of Virtual Power Plants for better control.
        \item Introduction of communication technologies within the micro-grids, enabling better information sharing and data handling.
        \item Implementation of ESaaS (Energy Storage as a Service )
          or Demand Side Management.
    \end{itemize}
    \item Market based ancillary solutions:
    \begin{itemize}
        \item Ensuring effective participation of prosumers by implementing proper regulatory framework and better incentives. The future electrical system being a decentralized one will involve a large number of producers as the price of generation (i.e. Solar , Wind) will be very less. Such a system will be very dynamic in terms of generation and demand. Hence a framework which is flexible enough to manage the fluctuations within the system and making sure that the producers are provided with proper stimulus (e.g. remuneration) is required to be present. 
        \item Making sure user to user energy trading is secure and reliable.Introduction of renewable sources will lead to the formation of small scale markets within which the producers of electricity will be able to trade it as a commodity via the TSO or without its involvement. For a better functioning of such type of energy markets, a certain set of rules are required which monitors the overall market.
        \item Inclusion of efficient internet and communication services. As there will be a need of proper data handling, a requirement of a stable network is created. Affordable and reliable services to ensure proper functioning of the market will be a prerequisite. 
    \end{itemize}
\end{enumerate}
%\\ 1. Technology based :
%   \\ a.  Energy Trading between micro-grids (user-user trading). 
%   \\b. Setting up of Virtual Power Plants for better control.
%   \\c. Introduction of communication technologies within the micro-grids, enabling better information sharing and data handling.
%   \\d.  Implementation of ESaaS (Energy Storage as a Service )
%          or Demand Side Management.
   
%  \\2. Market Based :
%  \\ a. Ensuring participation of prosumers by implementing proper regulatory framework and better incentives.
%  \\b. Making sure user to user energy trading is secure and reliable.
%  \\

\section{\textbf{Case 6:}}
The given total demand is:
\begin{equation}
    E_{total}=4000TWh
    \end {equation}
\begin{equation}
    capacity factor=\frac{E_{actual}}{P_{installed}*annual hours}
\end{equation}\label{eq1}
Assuming the Netherlands as our location, the wind energy capacity factors onshore are 0.35 (since it's a flat land and high wind speed zone) , and the offshore capacity factor is around 0.5.
Since the land area of the Netherlands is limited, only 30 \% of the wind load (2000 TWh) will be catered onshore, the remaining will be offshore
Hence substituting in \eqref{eq1} we get \\
P_{onshore}=196GW\\
P_{offshore}=320GW

In the Netherlands the annual sun hours are around 1000 hrs ( obtained from calculation performed in PV system course) hence the capacity factor for solar is 0.1143\\
P_{Solar}=1999.2 GW

It can be observed that the capacity of solar required in GW is very high, this is due to the lower capacity factor.

