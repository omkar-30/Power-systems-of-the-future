\justifying
{\textbf{Question 1:Planning criteria,operational criteria,remedial actions}}
    \begin{itemize}
        \item 1.1) The operational criteria employed by the TenneT operators i the control room is N-1, which means even if one component fails the system operation does not get affected.
        \item 1.2) Maintenance is not always possible as there might be an imbalance between the load and generation which could also lead to overloading of a particular line. To avoid any such scenario maintenance must be a planned activity.
        \item 1.3) N-1 is still followed during maintenance, inspite of the intuitive choice being N-2, a strict N-1 is followed.
        \item 1.4) VNB means outage planning , the unavailability of a section for maintenance.
        \item 1.5) The remedial actions to be taken while maintenance cannot be scheduled due to operational criteria are reducing the load on the line which has to undergo maintenance. The operator must also keep the contingency incase of an emergency.
        \item 1.6) There were many re-dispatch actions needed in Germany because of the integration of a large number of renewable sources such as wind and solar. The capacity of the wind varied from 4 MW on low wind days to 80 MW on high wind days.This required multiple dipatches to be made by the TSO.
        \item 1.7) Virtual power plants can also control loads and have access to power generation in a wider geography and hence in the case of emergencies, it can either control the load through demand response, or dispatch power from a plant far away.
        
        \end{itemize}

{\textbf{Question 2:Control}}
 \item The HV line is of 220 kV and the EHV line is of 380 kV . The topology followed is a meshed network.\\
 
{\textbf{Question 3:Control}}
\begin{itemize}
    \item The tasks of the TSOs are as follows:
    \begin{enumerate}
        \item Maintaining the grid so that its operation is robust and efficient.
        \item Maintaining balance between demand and supply.
        \item Facilitating a functioning, stable market.
    \end{enumerate}
    \item High volume of PV and wind sources in the grid might create some issues on the grid operation and maintenance. The impact of increasing the share of renewables is summarized below:
    \begin{enumerate}
        \item High operational uncertainty.
        \item Increase in market volatility. 
        \item Requirement of better security as the power flows increase in the grid.
        \item Requirement of accurate forecasting algorithms in order to predict the behaviour of the PV and wind sources.
    \end{enumerate}
    \item  The increasing share of the PV and wind in the system, changes the actions performed by the operators slightly. The operator has to make sure that, the overall grid security is maintained. The PV and wind must have a high priority and their curtailment is kept as a last option. The grid becomes more volatile which more flow based solutions to maintain grid stability.
    \item To reduce the operator's workload, efficient division of work within few teams of operators can be arranged. Better training can be provided to the operators in terms of grid operation and disturbance handling.
    One more aspect that can be looked in to is data-driven solutions. More specifically machine learning or through artificial intelligence. It can increase the accuracy and predictability of the entire system.
    \item For large scale RES integration the system has to be made more flexible in order to sustain the volatility of the sources. Improving the quality of data that is exchanged within the system, will enhance the communication network, which in turn increases the observability of the system. Introducing efficient fore-casting services or algorithms in order to maintain stability and security of supply.
\end{itemize}
{\textbf{Question 4:Real-time Operation}}\newline
During a solar eclipse, the generation via the solar plants reduces significantly. This causes the power gradient to increase drastically which causes imbalance in the system.
TSOs face challenge to regulate the capability of the entire inter-connected system. In order to  guarantee efficient real time operation, WAMS along with PMUs are required which gives better control over the system. The geographical location of every power reserve should be known in order to supply the loads.Strong co-ordination measures in between different TSOs are also required.\\

{\textbf{Question 5:Blackout Prevention}}
\begin{itemize}
    \item There are many events that can result in a power system blackout. The most important causes can be narrowed down to the following:
    \begin{enumerate}
        \item Natural Phenomena which the system is not able to withstand (e.g. Earthquake).
        \item Primary Equipment failure or a cascaded failure of equipment.
        \item Error in operation which can be a result of insufficient training of the operator. 
    \end{enumerate}\\
    \item As the power system now-a-days is evolving with inclusion of renewables, storage and smarter communication/control strategies, it experiences new phenomena every time. Hence it becomes difficult to have exact preventive measures for a blackout. Some are mentioned below:
    \begin{enumerate}
        \item Design the power system to intercept overloads. Inclusion of more electronic relays/fuses can be considered.
        \item Making the system more resilient to cyber-attacks.
        \item Including efficient primary and secondary control strategies to overcome frequency/voltage stabilities in the system.
    \end{enumerate}
    \item The impact of a failure can be reduced by islanding. Protection equipment which are able to intercept the fault/disturbance are needed to ensure the decoupling of the faulty area. This will also require coherent communication protocols which are more reliable in the event of a fault.
    \item The restoration is often started with a "black-start" procedure. After which the lines which are tripped are turned on consecutively. It is made sure that no personnel is working on the line that is restored. More trained operators are responsible for system restoration.
    \newpage
\end{itemize}
{\textbf{Question 6:Large disturbaces in Diemen}}
\item The first event that occured was the mechanical failure of the disconnector during a bus bar switch over. This was a technical fault. After this event the operators perceived the failure of secondary component inspite of the failure of a primary component.This further lead to cascading of events and causing a blackout.After the event, there was a retraining of the operators and a better understanding of faults.